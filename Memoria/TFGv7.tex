\documentclass[11pt,a4paper,twoside,pdf]{article}

% Paquetes (añade otros si los necesitas):
\usepackage{latexsym}
\usepackage[utf8x]{inputenc}
\usepackage{soul}
\usepackage{array}
\usepackage{amsmath}
\usepackage{amssymb}
\usepackage{marvosym}
\usepackage{epsfig}
\usepackage{graphics}
\usepackage{amsfonts}
\usepackage{xspace}
\usepackage{color}
\usepackage{booktabs}
\usepackage{xtab}
\usepackage[colorlinks=true,urlcolor=blue,linkcolor=blue,citecolor=blue]{hyperref}
\numberwithin{equation}{section}

% Fuente: palatino
\usepackage[sc]{mathpazo}
\linespread{1.05}

% TFG en inglés:
%\usepackage[english]{babel} 
%\addto\captionsenglish{\renewcommand{\chaptername}{}}

% TFG en español:
\usepackage[spanish,es-nodecimaldot,es-tabla,es-lcroman,es-nosectiondot,
            es-noindentfirst]{babel}
\renewcommand\spanishchaptername{}

% Formato de la página:
\usepackage{fancyhdr}
\usepackage[top=2.88cm,bottom=2.97cm,left=2.95cm,right=2.95cm]{geometry}
\setlength{\parskip}{0.1cm}

% Pon aquí tus definiciones:

\newcommand{\dis}{\displaystyle}
\sodef\an{}{.2em}{1em plus1em}{2em plus.1em minus.1em}

\begin{document}

% Portada %%%%%%%%%%%%%%%%%%%%%%%%%%%%%%%%%%%%%%%%%%%%%%%%%%%%%%%%%%%%%%%%%%%%%%

\pagestyle{empty}


\noindent
\begin{tabular}{r}
\includegraphics[width=8.8cm]{escudoUGRmonocromo.png} \\[-1.8ex]
\hspace{31mm}\vspace{-8mm}
\begin{tabular}{c}
\hline\\[-1ex]\hskip-2mm
{\bf Facultad de Ciencias}\hspace{18mm}
\end{tabular}
\end{tabular}

{\large
\vspace{30mm}
\hspace{25mm}
\begin{tabular}{l}
\an{GRADO EN F\'ISICA}
\end{tabular}

\vspace{45mm}
\hspace{25mm}
\begin{tabular}{l}
\an{TRABAJO FIN DE GRADO}
\\[1.5ex]
\an{\LARGE\bf T\'ITULO DEL TRABAJO}
\end{tabular}

\vfill
\hspace{25mm}
\begin{tabular}{l}
Presentado por:
\\
{\bf D./D$^{\rm a}$. Nombre y Apellidos}
\\[3ex]
Curso Académico 20\_\_/20\_\_
\end{tabular}
}


\newpage

\begin{center}

{\bf Resumen}
\bigskip

\begin{minipage}{0.8\linewidth}
bla bla bla bla bla bla bla bla bla bla bla bla bla bla bla
bla bla bla bla bla bla bla bla bla bla bla bla bla bla bla
bla bla bla bla bla bla bla bla bla bla bla bla bla bla bla 
bla bla bla bla bla bla bla bla bla bla bla bla bla bla bla
bla bla bla bla bla bla bla bla bla bla bla bla bla bla bla
\end{minipage}

\vfill

{\bf Abstract} 
\bigskip

\begin{minipage}{0.8\linewidth}
bla bla bla bla bla bla bla bla bla bla bla bla bla bla bla
bla bla bla bla bla bla bla bla bla bla bla bla bla bla bla
bla bla bla bla bla bla bla bla bla bla bla bla bla bla bla 
bla bla bla bla bla bla bla bla bla bla bla bla bla bla bla
bla bla bla bla bla bla bla bla bla bla bla bla bla bla bla
\end{minipage}

\vfill

\end{center}

% Indice %%%%%%%%%%%%%%%%%%%%%%%%%%%%%%%%%%%%%%%%%%%%%%%%%%%%%%%%%%%%%%%%%%%%%%%
\newpage

\tableofcontents

% Texto %%%%%%%%%%%%%%%%%%%%%%%%%%%%%%%%%%%%%%%%%%%%%%%%%%%%%%%%%%%%%%%%%%%%%%%%
\newpage

\pagestyle{fancy}
\fancyhead[RO,LE]{\leftmark}
\fancyhead[LO,RE]{\thepage}
\fancyfoot{}

\section{Introducción}

Ésta es la plantilla para escribir en \LaTeX\ los TFG de Física. En ella basta rellenar los nombres de la portada y las diferentes secciones. El índice se produce automáticamente. No debe alterarse el formato. Si el trabajo se escribe en inglés conviene descomentar en el encabezamiento del fichero {\sf tfg.tex} las líneas que aparecen bajo:
\begin{verbatim}
% TFG en inglés:
%\usepackage[english]{babel} 
%\addto\captionsenglish{\renewcommand{\chaptername}{}}
\end{verbatim}
y comentar las que aparecen bajo:
\begin{verbatim}
% TFG en español:
\usepackage[spanish,es-nodecimaldot,es-tabla,es-lcroman,es-nosectiondot,
            es-noindentfirst]{babel}
\renewcommand\spanishchaptername{}
\end{verbatim}

Las figuras y las tablas son objetos flotantes, pero si se desea puede forzarse su localización en la página, como se ha hecho con la siguiente figura:

\begin{figure}[h]
\centering
\includegraphics[width=8cm]{fig.png}				
\caption{Si no es de elaboración propia, debe especificarse la fuente \cite{PYTHIA}. \label{figura1}}
\end{figure}
\noindent

A continuación se han escrito algunas secciones y subsecciones, así como citas a ecuaciones, tablas, figuras y referencias para ilustrar su uso. El paginado y las nombres de las secciones aparecen automáticamente en la parte superior de cada página. 

Por si se desea reproducir este formato con otro procesador de textos: la fuente es palatino 11pt y los márgenes son 2.88 cm (superior), 2.97 cm (inferior), 2.95 cm (izquierdo y derecho).


\section{Una sección}

Scattering processes at high energy hadron colliders can be classified as either hard or soft. For hard processes, such as Higgs boson or high $p_{T}$ jet production, the rates and event properties can be predicted with good accuracy using perturbation theory.
For soft processes, such as elastic and diffractive processes, the rates and properties are dominated by non-perturbative QCD effects, which are less understood. In this section, these processes will be explained. The total cross section for proton-proton collisions is the sum of different contributions:
\begin{equation}
\sigma_{\rm tot}=\sigma_{\rm el}+\sigma_{\rm sd}+\sigma_{\rm dd}+\sigma_{\rm nd}
\end{equation}
It is important to notice that in a collider, such as the LHC or the Tevatron, there is a non-negligible probability that one single bunch crossing may produce several separate $pp$ collisions. This phenomenon is known as {\em pile up} and one has to consider it in realistic simulations.

\subsection{Una subsección}

Initially, two beams of particles are coming in towards each other. Normally each particle is characterized by a set of parton distributions, which defines the partonic substructure phenomenologically.

\begin{table}
\centering
\begin{tabular} {c|c}
Subprocess & $|{\cal{M}}|^2$ \\ \hline 
\\
$q_{i}q_{j}\rightarrow q_{i}q_{j}$ & $\dis\frac{4}{9}\dis\frac{s^2+u^2}{t^2}$ \\ \\
$q_{i}q_{i}\rightarrow q_{i}q_{i}$ & $\dis\frac{4}{9}\left(\dis\frac{s^2+u^2}{t^2}+\dis\frac{s^2+t^2}{u^2}\right)-\dis\frac{8}{27}\dis\frac{s^2}{ut}$ \\ \\
$q_{i}\overline{q}_j\rightarrow q_{i}\overline{q}_i$ &$\dis\frac{4}{9}\left(\dis\frac{s^2+u^2}{t^2}+\dis\frac{t^2+u^2}{s^2}\right)-\dis\frac{8}{27}\dis\frac{u^2}{st}$ \\ \\
$q_{i}\overline{q}_{i}\rightarrow gg$ & $\dis\frac{32}{27}\left(\dis\frac{u^2+t^2}{ut}\right)-\dis\frac{8}{3}\left(\dis\frac{u^2+t^2}{s^2}\right)$ 
\end{tabular}
\caption{Scattering amplitudes for several parton processes. \label{tabla1}}
\end{table}

The cross section for a process $ij\rightarrow k$ in which one parton of each incoming particle interact is given by \cite{PYTHIA}
\begin{equation}
\sigma_{ij\rightarrow k}=\int dx_{1} \int dx_{2}\, f_{i}^{1}(x_{1}) f_{j}^2(x_{2})\,\widehat{\sigma}_{ij\rightarrow k}
\end{equation}
Here $\widehat{\sigma}_{ij\rightarrow k}$ is the cross section for the hard partonic subprocess that has to be obtained using the machinery of Feynman diagrams. As a reminder, $f_{i}^{a}(x)$ are the parton-distribution functions
that encode the structure of the proton. Some of these scattering amplitudes squared are in the table \ref{tabla1}, where $s,t,u$ are the Mandelstam variables, depending on the four-momenta of the scattering particles in a Lorentz-invariant way. They are used for scattering processes of two particles to two particles and are defined by:
\begin{align}
s&=(p_1+p_2)^2=(p_3+p_4)^2 \\
t&=(p_1-p_3)^2=(p_2-p_4)^2 \\
u&=(p_1-p_4)^2=(p_2-p_3)^2
\end{align}

\section{Conclusiones}

Esta sección no debería faltar en todo TFG. Después van las referencias que pueden añadirse en la misma página o en una nueva, como se ha hecho aquí. Hay ejemplos de cómo deben citarse artículos \cite{PYTHIA}, actas de congresos \cite{conferencia}, charlas \cite{charla}, libros \cite{libro}, tesis doctorales \cite{tesis} y páginas web \cite{web}.

% Referencias %%%%%%%%%%%%%%%%%%%%%%%%%%%%%%%%%%%%%%%%%%%%%%%%%%%%%%%%%%%%%%%%%
\newpage

\addcontentsline{toc}{section}{Referencias} % Elige según idioma
%\addcontentsline{toc}{section}{References} % Elige según idioma

\begin{thebibliography}{100}

\bibitem{PYTHIA}
  T.~Sjostrand, S.~Mrenna and P.~Z.~Skands, \\
  {\em PYTHIA 6.4 Physics and Manual}, \\
  JHEP {\bf 0605} (2006) 026  [hep-ph/0603175].
  
\bibitem{conferencia}
  F.~Wilczek, \\
  {\em A long view of particle physics}, \\
  Proceedings of the 25th Solvay Conference on Physics, p. 210-249, \\  
  Brussels, Belgium, October 19-25, 2011.

\bibitem{charla}
  D.~Gross, \\
  {\em Quantum Field Theory: Past, Present and Future}, \\
  Talk at the Conference in Honour of the 90th Birthday of Freeman Dyson, \\
  Institute of Advanced Studies, Singapore, August 26-29, 2013.

\bibitem{libro}
  M.~E. Peskin and D.~V. Schroeder, \\
  {\em An Introduction to Quantum Field Theory}, \\
  Addison-Wesley, 1995.
  
\bibitem{tesis}
  M.~R.~Chala, \\
  {\em Collider Signatures of a Non-Standard Higgs Sector}, \\
  PhD Thesis, Universidad de Granada, 2014.

\bibitem{web}
 Particle Physics News and Resources, \\
 \href{http://www.interactions.org/}{http://www.interactions.org/}
 
\end{thebibliography}

\end{document}
